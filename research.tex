\section*{Research}

During my time at MIT, I have conducted research in a number of different topics in Theoretical Computer Science,
including Sublinear algorithms, local algorithms, computational geometry \& origami theory, and distributed algorithms.
\todo{remove distributed algo, or mention DDD}

\subsection*{Sublinear Algorithms}
My first exposure through research was through a graduate course I attended my freshman year titled ''Randomness and Computation''.
Our instructor was Ronitt Rubinfeld, and being the only undergraduate in a small class, I got to know her quite well.
I was invited to join in group meetings with her PhD students, who were working on property testing on graphs.
Having considered $ \mathcal{O}(n) $ to be the gold standard of complexity all my life,
I was very excited to learn more about sublinear algorithms.

\paragraph{Counting Subgraphs.}
Our work (published in Algorithmica \cite{stars}) was focused on estimating the number of star subgraphs in a given graph.
We introduced a new model in graph property testing, that allowed for a random edge to be sampled.
Most prior reseach had only considered random vertex, and random neighbor queries.
However, the ability to sample an uniformly random edge is very natural,
since it is equivalent to sampling a random element from an adjacency list representation of a graph.
We were able to show that this model is more powerful than the old vertex sampling models,
by improving the bounds presented in \cite{old_stars}.

\paragraph{Local Access Implementations.}
More recently, I have started working on a new research project (also under Ronitt's tutelage).
in the field of \emph{local-access implementations} of huge random objects.
This is a relatively new area, pioneered in \cite{huge, huge_old}, and further developed in \cite{sparse, reut}.
This model considers huge random objects such as exponentially sized random graphs.
Classically, such objects are fully generated, before being studied.
However, in the context of algorithms with sub-linear query complexity, it seems wasteful to generate the entire random object.
Local-access implementations aim to provide oracle access to thes random objects,
without having a large pre-processing overhead.
%For instance, we can provide access to the adjacency list of a random graph in the form of neighbor queries.
%In this setting, the oracle can generate portions of the graph \emph{on-the-fly} from the correct distribution,
%with some overhead (usually $ \mathcal{O}(poly\log n) $ time, space, and random bits).

We have obtained some new and interesting results in this area.
particularly for undirected random graphs with independent edge probabilities.
We provide the first implementations of a $\func{random-neighbor}$ query,
which would allow efficient implementation of random walk based algorithms.
Our implementation is also the first to support dense graphs,
and unlike many past results, we can genrate the entire graph without introducing inconsistencies.
%Whereas most previous results \cite{huge, sparse} have supported a limited number of queries on sparse graphs,
%our generators can be run to completion without introducing any inconsistencies (similar to \cite{reut}).
Specifically, we show that our algorithm can be used to efficiently implement Erdos-Renyi graphs, and the Stochastic Block Model.
We also design generators for graphs in Kleinberg's small world model,
where edge probabilities depend on a 2-dimensional distance metric.

Currently, I am working on several other classes of random objects.
In the realm of graph theory, I have been considering implementations of Geometric and Hyperbolic Random Graphs\todo{cite}.
We also have some partial results for locally generating uniformly random domino tilings,
\todo{dimer model physics etc \ldots}
and obtaining uniform samples from Dyck languages.

\subsection*{Computational Geometry}

Over the past year, I have also had the opportunity to work on several problems 
in origami and computational geometry with Erik Demaine.

\paragraph{Origami.}
One of the results from this class was a result on common unfoldings.
Our paper\cite{common}, published in CCCG, outlines the construction of a development,
that can fold into twelve different convex polyhedra (the list is now slightly longer),
in five different combinatorial classes.
Most of the past results in the field \cite{box2, box2.5, box3, cube_tet, jz}
only result in a small number of unfoldings.
Additionally, our construction presents the first non-trivial \emph{uncountable family} of unfoldings.

Another direction I worked on was concerning the origami construction of extruded surfaces.
Here, I developed a construction for arbitrary extruded orthogonal graphs,
with $ \mathcal{O}(2+\varepsilon) $ optimal paper usage, under some suitable assumptions.
We also formulated a technique for constructing arbitrary extruded polygons,
where convex polygons can be made with $ \mathcal{O}(1+1/n) $ optimal paper usage.
\todo[inline]{Compare to past results}

\paragraph{Auxetic Linkages.}
One of my ongoing projects is on the design of 2-dimensional auxetic linkages.
Specifically, I am interested in inhomogeneous bistable auxetic linkages.
Bistable auxetic linkages that have exactly two possible configurations,
and it is not possible to continuously deform them into each other.
Generally, the final state of the linkage will have a larger area.

Existing literature focuses on periodic linkages that have homogenous expansion properties.
I am investigating an aperiodic linkage that has varying amounts of expansion in different locations.
This implies that the final state is non-Euclidean, and I have constructed some physical models,
that start out from a flat sheet, and unfold into a 3-dimensional surface.
\todo[inline]{Pictures}

