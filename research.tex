\section{Research}
\label{sec:research}

During my time at MIT, I have conducted research in a number of different topics% in Theoretical Computer Science,
including sublinear algorithms, local algorithms, computational geometry and origami theory.

\subsection*{Sublinear Algorithms}
My first exposure to research was in freshman year through a graduate course titled ''Randomness and Computation''.
Our instructor was Ronitt Rubinfeld, and being the only undergraduate student in a small class, I got to know her quite well.
I was invited to join in group meetings with her PhD students, who were working on graph property testing.
Having considered $ \mathcal{O}(n) $ to be the gold standard of complexity all my life,
I was very excited to learn more about sublinear algorithms.

\paragraph{Counting Subgraphs.}
Our work (published in Algorithmica \cite{stars}) was focused on estimating the number of star subgraphs in a given graph.
We introduced a new model in graph property testing, which allowed for a random edge to be sampled.
Most prior research had only considered random vertex, and random neighbor queries.
However, the ability to sample a uniformly random edge is very natural,
since it is equivalent to sampling a random element from an adjacency list representation of a graph.
We were able to show that this model is more powerful than the old vertex sampling models,
by improving the bounds presented in \cite{old_stars}.

\paragraph{Local Access Implementations.}
More recently, I have started working on a new research project concerning \emph{local-access implementations} of huge random objects,
a field that was pioneered in \cite{huge, huge_old}, and further developed in \cite{sparse, reut}.
This model considers huge random objects such as exponentially sized random graphs.
Classically, such objects are fully generated, before being studied.
However, in the context of sub-linear query complexity, it is wasteful to generate the entire object.
Local-access implementations aim to provide oracle access to these random objects, while avoiding a large pre-processing overhead.

We have obtained some new and interesting results in this area --
particularly for undirected random graphs with independent edge probabilities.
We provide the first implementations of $\func{random-neighbor}$ queries.
Our implementation also introduces support dense graphs, and unlike many past results,
we can generate the entire graph without introducing inconsistencies.
%Whereas most previous results \cite{huge, sparse} have supported a limited number of queries on sparse graphs,
%our generators can be run to completion without introducing any inconsistencies (similar to \cite{reut}).
Specifically, we show that our algorithm can be used to efficiently implement $G(n,p)$ graphs, and the Stochastic Block Model.
We also design generators for graphs in Kleinberg's small world model,
where edge probabilities depend on a 2-dimensional distance metric.
%Our paper was submitted to \emph{STOC 2018}, and is currently in review.

Currently, I am working on implementations of Geometric and Hyperbolic Random Graphs.
We have also investigated the local generation of uniformly random domino tilings
(which has applications in the dimer model), and uniform samples from Dyck languages.

\subsection*{Computational Geometry}

I have also worked on several problems in origami and computational geometry with Erik Demaine.
%initiated in the class titled \emph{Geometric Folding Algorithms: Linkages, Origami, Polyhedra}.

\paragraph{Origami.}
One of the results from this class was a result on common unfoldings; 2D shapes that can fold into multiple different polyhedra.
Our paper \cite{common}, published in CCCG, outlines the construction of a development that can fold into twelve different convex polyhedra.
Most of the past results in the field \cite{box2, box2.5, box3, cube_tet, jz} only result in a small number of unfoldings.
Our construction also presents the first non-trivial \emph{uncountable family} of unfoldings.

I have also worked on computational origami design, introducing a novel design framework which uses time-evolving cross-sections.
This results in a construction for arbitrary orthogonal terrains, with $2$-optimal paper usage.
We also formulate a technique for constructing arbitrary extruded polygons,
where convex polygons can be made with $ 1+\mathcal{O}(1/n) $ optimal paper usage.

\paragraph{Auxetic Linkages.}
One of my ongoing projects is on the design of 2-dimensional auxetic linkages.
These are periodic linkages with an expansive deformation \cite{aux0,aux1}.
Specifically, I am interested in inhomogeneous bistable auxetic linkages;
linkages with exactly two configurations, that are isolated in state space.
While existing literature focuses on periodic linkages with \emph{continuous homogenous expansion},
I am investigating \emph{aperiodic} linkages that have \emph{position dependent} expansion coefficients.
This results in a non-Euclidean final state, and allows us to approximate any desired 3D surface.
I have constructed physical models that start out as a flat sheet, and unfold into a 3D surface.
%\todo[inline]{Pictures}
