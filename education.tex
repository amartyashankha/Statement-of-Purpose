\section*{Education}
I grew up in a college campus in India. I have always been interested in Mathematics and making things. 
My introduction to computer science started following a week long workshop in $8^{th}$ grade,
where I was introduced to C programing.
That one could build powerful solutions to problems immediately resonated with me.
My interest led me to explore more and to compete for the National Informatics Olympiad in my 9th grade.
Subsequently I represented India thrice in the International Informatics Olympiad and got medals twice.
The training sessions and our instructors in the training camp introduced me
to the wonderful world of algorithms and data structures.

Unsurprisingly, when I started college, I was drawn towards theoretical Computer Science.
I had the opportunity to take several upper level and graduate courses in this domain
starting right from my freshman year, which boosted my interest.
Through the years at MIT, I have actively pursued some exciting research topics.
My primary research exposure has been in the fields of Sublinear algorithms
under the guidance of Ronitt Rubinfeld, and computational geometry under Erik Demaine.

I have also sought out many teaching related activities in MIT which have been a satisfying experience for me.
I have been a TA for the second level undergraduate algorithms course (6.046) three times, starting from my sophomore year.
I also organized and taught a month long independent algorithms crash course for MIT students.
I completely conceived and organized a programming, simulation and robotics workshop for high school students
in my hometown in India, as a joint venture between MIT and IIT Kharagpur (Indian Institute of Technology).

I am interested in a career in academia that will allow me to pursue my interests in research and teaching.
Pursuing a PhD will enable me to delve more intensely in research in theoretical computer science,
and prepare me for my academic career.
