\section*{Education}

My introduction to computer science started following a week long workshop high school, where I learned how to write C.
and was further developed when I became involved with competitive programming.
I ended up attending IOI (International Olympiad in Informatics) as part of the Indian team for three years,
and I even won a couple of medals.
These experiences introduced me to the wonderful world of algorithms and data-structures.

Unsurprisingly, once I was in college, I was drawn towards theoretical Computer Science.
Through the five years of my ungergraduate and masters degree at MIT,
I have had the opportunity to participate in exciting reseach areas. \todo{What areas?}
My main research interests have been in the fields of Sublinear algorithms under the guidance of Ronitt Rubinfeld,
and computational geometry under Erik Demaine.

In my time at MIT, I have also had the pleasure of participating in many teaching related activities.
I have been a TA for the second level undergraduate algorithms course three times, starting my sophomore year.
I also organized and taught a month long independent algorithms crash course for MIT students.
Outside of MIT, I started a programming, simulation and robotics workshop for high school students in my hometown,
as a joint venture between MIT, and IIT Kharagpur (Indian Institute of Tehnology).

Based on my interests in research and teaching, I am interested in a career in academia.
In sight of this goal, I would like to pursue a PhD centered around theoretical computer science,
